\documentclass{42-pt}
\newcommand\qdsh{\texttt{42sh}}



%******************************************************************************%
%                                                                              %
%                               Prologue                                       %
%                                                                              %
%******************************************************************************%

\begin{document}
\title{A Norma}
\subtitle{Versão 4.1}

\summary
{
    Este documento descreve o padrão aplicável (Norma) na 42. Um padrão de
	programação define um conjunto de regras a seguir ao escrever um código.
    A Norma aplica-se a todos os projetos C dentro do Common Core por padrão, e
    para qualquer projeto onde é especificado.
}

\maketitle

\tableofcontents



%******************************************************************************%
%                                                                              %
%                                 Introdução                                   %
%                                                                              %
%******************************************************************************%
\chapter{Introdução}

A \texttt{norminette} é um código Python e open source que verifica a conformidade
do seu código-fonte com a Norma. Ela verifica muitas restrições da Norma, mas
não todas (por exemplo, restrições subjetivas). A menos que haja regulamentos
locais específicos em seu campus, a \texttt{norminette} prevalece durante
avaliações nos itens controlados. Nas páginas a seguir, as regras que não são
verificadas pela \texttt{norminette} são marcadas com \textit{(*)}, e podem levar
à reprovação do projeto (usando a flag da Norma) se descobertas pelo
avaliador durante uma revisão de código.\\

Seu repositório está disponível em https://github.com/42School/norminette.\\

Pull requests, sugestões e issues são bem-vindos!

\newpage


%******************************************************************************%
%                                                                              %
%                                   Explicações Pedago                         %
%                                                                              %
%******************************************************************************%
    \chapter{Por quê ?}

    A Norma foi cuidadosamente elaborada para suprir diversas necessidades
    pedagógicas. Aqui estão alguns dos motivos mais importantes por trás das
    escolhas abaixo:
    \begin{itemize}

	\item Sequenciamento: programar implica dividir uma tarefa grande e
	  complexa em uma série de instruções elementares. Todas essas instruções
	  vão ser executadas em sequência: uma após a outra. Um iniciante, ao
      começar a criar software, precisa de uma arquitetura simples e
	  clara para seu projeto, tendo o entendimento completo de todas as
      instruções individuais e da exata ordem de execução. Sintaxes de
      linguagens crípticas que aparentam executar múltiplas
	  instruções ao mesmo tempo são confusas, funções que buscam abordar
      múltiplas tarefas misturadas na mesma porção de código são fontes
      de erros. \\
	  A Norma pede que você escreva trechos simples de código cujas tarefas
	  possam ser entendidas e verificadas facilmente, em que a sequência de
      execução das instruções não deixa dúvidas. Por este motivo que há o
      limite máximo de 25 linhas por função, e o porquê de \texttt{for},
      \texttt{do .. while}, ou ternários serem proibidos.


    \item Estética: enquanto se relaciona com seus colegas durante o processo
      natural de aprendizado entre pares, e também durante as avaliações
      entre pares, você não quer gastar tempo decifrando o código deles,
      mas falar diretamente sobre a lógica daquele trecho de código.\\ A
      Norma pede por uma estética específica, provendo instruções para
      nomear funções e variáveis, indentação, utilização das chaves,
      tabulações e espaços em diversos lugares... . Isso vai permitir que
      você olhe brevemente para o código de outros e o ache familiar, podendo
      ir direto ao assunto, ao invés de gastar tempo lendo o código antes
      de entendê-lo. A Norma também se caracteriza como uma marca
      registrada. Como parte da comunidade 42, você vai poder reconhecer
      código escrito por outro cadete ou alumni da 42 quando estiver no
      mercado de trabalho.
    

    \item Visão de longo prazo: esforçar-se para escrever um código compreensível
      é a melhor maneira de administrá-lo. Toda vez que alguém, incluindo
      você, precisar consertar um bug ou adicionar uma nova
      funcionalidade, não será necessário gastar tempo tentando entender
      o funcionamento se você escreveu seu código da maneira correta. Isso
      vai evitar situações em que trechos de código deixam de ser
      atualizados apenas por tomarem tempo, o que vai fazer a diferença
      ao falarmos sobre ter um produto bem sucedido no mercado. Quanto
      mais cedo aprender, melhor.


    \item Referências: você pode pensar que algumas, ou todas, as regras
      inclusas na Norma são arbitrárias, mas nós pensamos cuidadosamente e
      pesquisamos como elaborá-la. Nós encorajamos fortemente que você
      pesquise o porquê de funções precisarem ser curtas e possuir apenas
      uma tarefa, o porquê de nomes de variáveis precisarem ser
      compreensíveis, o porquê de linhas não poderem extrapolar o limite
      de 80 colunas de largura, o porquê de uma função não poder receber
      vários parâmetros, o porquê de comentários serem úteis, etc.


    \end{itemize}


\newpage

%******************************************************************************%
%                                                                              %
%                                  A Norma                                     %
%                                                                              %
%******************************************************************************%
\chapter{A Norma}


%******************************************************************************%
%                            Convenções de nomeação                            %
%******************************************************************************%
    \section{Denominação}

        \begin{itemize}

            \item O nome de um struct deve começar por
                \texttt{s\_}.

            \item O nome de um typedef deve começar por
                \texttt{t\_}.

            \item O nome de um union deve começar por \texttt{u\_}.

            \item O nome de um enum deve começar por \texttt{e\_}.

            \item O nome de uma variável global deve começar por \texttt{g\_}.
            
           \item Identificadores, como nomes de variáveis, funções e tipos definidos pelo usuário,
                só podem conter letras minúsculas, dígitos e '\_' (snake\_case). Nenhuma letra maiúscula é permitida.

            \item Nomes de arquivos e diretórios só podem conter letras minúsculas, dígitos e
                '\_' (snake\_case).

            \item Caracteres que não fazem parte da tabela ASCII padrão são proibidos, exceto dentro de strings literais e caracteres.

            \item \textit{(*)} Todos os nomes de identificadores (funções, tipos,
                variáveis, etc.) devem ser explícitos, ou mnemônicos,
                devem ser legíveis em inglês, com cada palavra separada por um underscore.
                Isso se aplica a macros, nomes de arquivos e diretórios também.

            \item O uso de variáveis globais que não são marcadas como const ou static é
                proibido e é considerado um erro de norma, a menos que o projeto as permita explicitamente.

            \item O arquivo deve compilar. Não se espera que um arquivo que não compila
                passe na Norma.
        \end{itemize}
\newpage

%******************************************************************************%
%                                 Formatação                                   %
%******************************************************************************%
    \section{Formatação}

            \begin{itemize}

            \item Cada função deve ter no máximo 25 linhas, sem contar as próprias chaves da função.

            \item Cada linha deve ter no máximo 80 colunas de largura, incluindo comentários.
            Atenção: uma tabulação não conta como uma única coluna, mas como o número de espaços que ela representa.

            \item Funções devem ser separadas por uma linha vazia. Comentários ou instruções de pré-processador
            podem ser inseridos entre funções. Pelo menos uma linha vazia deve existir.

            \item Você deve indentar seu código com tabulações de 4 caracteres.
            Isso não é o mesmo que 4 espaços, estamos falando de tabulações reais (caractere ASCII número 9).
            Verifique se seu editor de código está corretamente configurado para obter uma indentação visual adequada
            que será validada pela \texttt{norminette}.

            \item Blocos dentro de chaves devem ser indentados. As chaves ficam sozinhas em sua própria linha,
            exceto na declaração de struct, enum, union.

            \item Uma linha vazia deve estar realmente vazia: sem espaços ou tabulações.

            \item Uma linha nunca pode terminar com espaços ou tabulações.
            
            \item Nunca pode haver duas linhas vazias consecutivas.
            Nunca pode haver dois espaços consecutivos.

            \item Declarações devem estar no início de uma função.

            \item Todos os nomes de variáveis devem ser indentados na mesma
            coluna em seu escopo. Nota: os tipos já são indentados pelo bloco que os contém.

            \item Os asteriscos que acompanham ponteiros devem estar colados
            ao nome da variável.

            \item Apenas uma declaração de variável por linha.

            \item Declaração e inicialização não podem estar
            na mesma linha, exceto para variáveis globais (quando permitido),
            variáveis estáticas e constantes.

            \item Em uma função, você deve colocar uma linha vazia entre
            as declarações de variáveis e o restante da função.
            Nenhuma outra linha vazia é permitida dentro de uma função.

            \item Apenas uma instrução ou estrutura de controle por linha é permitida. Ex.: Atribuição em
            uma estrutura de controle é proibida, duas ou mais atribuições na mesma linha são proibidas,
            uma nova linha é necessária ao final de uma estrutura de controle, ... .

            \item Uma instrução ou estrutura de controle pode ser dividida em várias linhas quando necessário.
            As linhas seguintes devem ser indentadas em relação à primeira linha,
            espaços naturais devem ser usados para cortar a linha, e, se aplicável, operadores devem estar
            no início da nova linha e não no final da anterior.

            \item A menos que seja o final de uma linha, cada vírgula ou ponto e vírgula
            deve ser seguido por um espaço.

            \item Cada operador ou operando deve ser separado por um
            - e apenas um - espaço.

            \item Cada palavra-chave do C deve ser seguida por um espaço, exceto por
            palavras-chave de tipos (como int, char, float, etc.),
            assim como sizeof.

            \item Estruturas de controle (if, while..) devem usar chaves, a menos que contenham uma única
            instrução em uma única linha.

            \end{itemize}

\vspace{1cm}

            Exemplo geral:
            \begin{42ccode}
int             g_global;
typedef struct  s_struct
{
    char    *my_string;
    int     i;
}               t_struct;
struct          s_other_struct;

int     main(void)
{
    int     i;
    char    c;

    return (i);
}
            \end{42ccode}
            \newpage

%******************************************************************************%
%                              Parâmetros de função                             %
%******************************************************************************%
    \section{Funções}

        \begin{itemize}

            \item Uma função pode ter até 4 parâmetros definidos no máximo.

            \item Uma função que não tem argumentos deve ser
                explicitamente prototipada com a palavra "void" como o
                argumento.

            \item Parâmetros em protótipos de funções devem ser nomeados.

            \item Cada função deve ser separada da próxima por
                uma linha vazia.

            \item Você não pode declarar mais de 5 variáveis por função.

            \item O retorno de uma função deve estar entre parênteses, a menos
                que a função retorne nada.

            \item Cada função deve ter uma tabulação única entre seu
                tipo de retorno e seu nome.

        \end{itemize}

\vspace{1cm}

            \begin{42ccode}
int my_func(int arg1, char arg2, char *arg3)
{
    return (my_val);
}

int func2(void)
{
    return ;
}
            \end{42ccode}

        \newpage


%******************************************************************************%
%                        Typedef, struct, enum e union                         %
%******************************************************************************%
    \section{Typedef, struct, enum e union}

        \begin{itemize}

        \item Como outras palavras-chave do C, adicione um espaço entre ``struct'' e o nome
          ao declarar uma struct. O mesmo se aplica para enum e union.

        \item Ao declarar uma variável do tipo struct, aplique a indentação usual para o nome
          da variável. O mesmo se aplica para enum e union.

        \item Dentro das chaves da struct, enum, union, as regras regulares de indentação
          se aplicam, como em qualquer outro bloco.

        \item Como outras palavras-chave do C, adicione um espaço após ``typedef'',
          e aplique a indentação regular para o novo nome definido.

        \item Você deve indentar todos os nomes das estruturas na mesma coluna para seu escopo.

        \item Você não pode declarar uma estrutura em um arquivo .c.

        \end{itemize}
        \newpage


%******************************************************************************%
%                                   Headers                                    %
%******************************************************************************%
    \section{Headers - ou arquivos de inclusão}

      \begin{itemize}

          \item \textit{(*)} Os elementos permitidos em um arquivo header são:
              inclusões de headers (sistema ou não), declarações, defines,
              protótipos e macros.

          \item Todas as inclusões devem estar no início do arquivo.

          \item Você não pode incluir um arquivo C em um header ou em outro arquivo C.

          \item Arquivos header devem ser protegidos contra inclusões duplas. Se o arquivo for
          \texttt{ft\_foo.h}, sua macro de proteção deve ser \texttt{FT\_FOO\_H}.

          \item \textit{(*)} A inclusão de headers não utilizados é proibida.

          \item A inclusão de headers pode ser justificada no arquivo .c e no próprio arquivo .h
            usando comentários.

      \end{itemize}

\vspace{1cm}

      \begin{42ccode}
#ifndef FT_HEADER_H
# define FT_HEADER_H
# include <stdlib.h>
# include <stdio.h>
# define FOO "bar"

int     g_variable;
struct  s_struct;

#endif
        \end{42ccode}
        \newpage

%******************************************************************************%
%                                 O Header da 42                               %
%******************************************************************************%

  \section{O header da 42 - ou comece um arquivo com estilo}

    \begin{itemize}

    \item Todo arquivo .c e .h deve começar imediatamente com o header padrão da 42:
      um comentário multilinha com um formato especial incluindo informações úteis. O
      header padrão está disponível nos computadores dos clusters para vários
      editores de texto (emacs: usando \texttt{C-c C-h}, vim usando \texttt{:Stdheader} ou
      \texttt{F1}, etc...).

    \item \textit{(*)} O header da 42 deve conter várias informações atualizadas, incluindo o
      criador com login e e-mail estudantil (@student.campus), a data de criação,
      o login e a data da última atualização. Cada vez que o arquivo for salvo no disco,
      as informações devem ser atualizadas automaticamente.

    \end{itemize}
    \info{
      O header padrão pode não estar automaticamente configurado com suas informações pessoais.
      Você pode precisar alterá-lo para seguir a regra anterior.
      }
    
    \newpage     
                
%******************************************************************************%
%                           Macros e pré-processadores                         %
%******************************************************************************%
    \section{Macros e Pré-processadores}

      \begin{itemize}

          \item \textit{(*)} Constantes de pré-processador (ou \#define) que você criar devem ser usadas
              apenas para valores literais e constantes.
          \item \textit{(*)} Todo \#define criado para burlar a norma e/ou ofuscar
              o código é proibido.
          \item \textit{(*)} Você pode usar macros disponíveis em bibliotecas padrão, somente
              se estas forem permitidas no escopo do projeto em questão.
          \item Macros multilinha são proibidas.
          \item Nomes de macros devem estar todos em maiúsculas.
          \item Você deve indentar diretivas de pré-processador dentro de blocos \#if, \#ifdef
              ou \#ifndef.
          \item Instruções de pré-processador são proibidas fora do escopo global.

      \end{itemize}
      \newpage


%******************************************************************************%
%                              Coisas proibidas!                               %
%******************************************************************************%
    \section{Coisas proibidas!}

        \begin{itemize}

            \item Você não tem permissão para usar:

                \begin{itemize}

                    \item for
                    \item do...while
                    \item switch
                    \item case
                    \item goto

                \end{itemize}

            \item Operadores ternários como `?'.

            \item VLAs - Arrays de comprimento variável.

            \item Tipo implícito em declarações variáveis.

        \end{itemize}

\vspace{1cm}

        \begin{42ccode}
    int main(int argc, char **argv)
    {
        int     i;
        char    string[argc]; // This is a VLA

        i = argc > 5 ? 0 : 1 // Ternary
    }
        \end{42ccode}
        \newpage

%******************************************************************************%
%                                   Comments                                   %
%******************************************************************************%
    \section{Comentários}

        \begin{itemize}

        \item Comentários não podem estar dentro do corpo das funções.
            Comentários devem estar no final de uma linha ou em sua própria linha.

        \item \textit{(*)} Seus comentários devem estar em inglês e ser úteis.

        \item \textit{(*)} Um comentário não pode justificar a criação de uma função genérica/faz-tudo ou ruim.

        \end{itemize}

        \warn{
            Uma função genérica/faz-tudo ou ruim geralmente possui nomes que não são
            explícitos, como f1, f2... para funções e a, b, c,... para nomes de variáveis.
            Uma função cujo único objetivo é evitar a norma, sem um propósito
            lógico único, também é considerada uma função ruim.
            Lembre-se de que é desejável ter funções claras e legíveis que realizem
            uma tarefa clara e simples. Evite qualquer técnica de ofuscação de código,
            como one-liner, ... .
        }
        \newpage


%******************************************************************************%
%                                    Files                                     %
%******************************************************************************%
    \section{Arquivos}

        \begin{itemize}

            \item Você não pode incluir um arquivo .c em um arquivo .c.

            \item Você não pode ter mais de 5 definições de função em um arquivo .c.

        \end{itemize}
        \newpage


%******************************************************************************%
%                                   Makefile                                   %
%******************************************************************************%
    \section{Makefile}

            Makefiles não são verificados pela \texttt{norminette} e devem ser checados durante a avaliação pelo
            estudante quando solicitado pelas diretrizes de avaliação. A menos que haja instruções específicas, as seguintes regras
            se aplicam aos Makefiles:
            \begin{itemize}

                \item As regras \textit{\$(NAME)}, \textit{clean}, \textit{fclean}, \textit{re} e \textit{all}
                  são obrigatórias. A regra \textit{all} deve ser a padrão e executada ao digitar apenas \texttt{make}.

                \item Se o makefile fizer relink quando não for necessário, o projeto será considerado
                  não funcional.

                \item No caso de um projeto com múltiplos binários, além das regras acima, você deve ter uma regra para cada binário (ex: \$(NAME\_1), \$(NAME\_2), ...).
                  A regra ``all'' irá compilar todos os binários, utilizando a regra de cada binário.

                \item No caso de um projeto que utiliza uma função de uma biblioteca não do sistema
                  (ex.: \texttt{libft}) que existe junto ao seu código fonte, seu makefile deve compilar
                  essa biblioteca automaticamente.

                \item Todos os arquivos fonte necessários para compilar seu projeto devem
                  ser explicitamente nomeados no seu Makefile. Ex.: nada de ``*.c'', nada de ``*.o'', etc ...

            \end{itemize}


\end{document}
%******************************************************************************%
